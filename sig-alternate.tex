% This is "sig-alternate.tex" V2.0 May 2012
% This file should be compiled with V2.5 of "sig-alternate.cls" May 2012
%
% This example file demonstrates the use of the 'sig-alternate.cls'
% V2.5 LaTeX2e document class file. It is for those submitting
% articles to ACM Conference Proceedings WHO DO NOT WISH TO
% STRICTLY ADHERE TO THE SIGS (PUBS-BOARD-ENDORSED) STYLE.
% The 'sig-alternate.cls' file will produce a similar-looking,
% albeit, 'tighter' paper resulting in, invariably, fewer pages.
%
% ----------------------------------------------------------------------------------------------------------------
% This .tex file (and associated .cls V2.5) produces:
%       1) The Permission Statement
%       2) The Conference (location) Info information
%       3) The Copyright Line with ACM data
%       4) NO page numbers
%
% as against the acm_proc_article-sp.cls file which
% DOES NOT produce 1) thru' 3) above.
%
% Using 'sig-alternate.cls' you have control, however, from within
% the source .tex file, over both the CopyrightYear
% (defaulted to 200X) and the ACM Copyright Data
% (defaulted to X-XXXXX-XX-X/XX/XX).
% e.g.
% \CopyrightYear{2007} will cause 2007 to appear in the copyright line.
% \crdata{0-12345-67-8/90/12} will cause 0-12345-67-8/90/12 to appear in the copyright line.
%
% ---------------------------------------------------------------------------------------------------------------
% This .tex source is an example which *does* use
% the .bib file (from which the .bbl file % is produced).
% REMEMBER HOWEVER: After having produced the .bbl file,
% and prior to final submission, you *NEED* to 'insert'
% your .bbl file into your source .tex file so as to provide
% ONE 'self-contained' source file.
%
% ================= IF YOU HAVE QUESTIONS =======================
% Questions regarding the SIGS styles, SIGS policies and
% procedures, Conferences etc. should be sent to
% Adrienne Griscti (griscti@acm.org)
%
% Technical questions _only_ to
% Gerald Murray (murray@hq.acm.org)
% ===============================================================
%
% For tracking purposes - this is V2.0 - May 2012

\documentclass{sig-alternate}

\begin{document}
%
% --- Author Metadata here ---
%\conferenceinfo{WOODSTOCK}{'97 El Paso, Texas USA}
%\CopyrightYear{2007} % Allows default copyright year (20XX) to be over-ridden - IF NEED BE.
%\crdata{0-12345-67-8/90/01}  % Allows default copyright data (0-89791-88-6/97/05) to be over-ridden - IF NEED BE.
% --- End of Author Metadata ---

\title{{\ttlit To Saindo} - Sítio da Web de ajuda a universitários para encontrar carona.
%\titlenote{(Produces the permission block, and
%copyright information). For use with
%SIG-ALTERNATE.CLS. Supported by ACM.}
}
\subtitle{
%\titlenote{A full version of this paper is available as
%\textit{Author's Guide to Preparing ACM SIG Proceedings Using
%\LaTeX$2_\epsilon$\ and BibTeX} at
%\texttt{www.acm.org/eaddress.htm}}
}
%
% You need the command \numberofauthors to handle the 'placement
% and alignment' of the authors beneath the title.
%
% For aesthetic reasons, we recommend 'three authors at a time'
% i.e. three 'name/affiliation blocks' be placed beneath the title.
%
% NOTE: You are NOT restricted in how many 'rows' of
% "name/affiliations" may appear. We just ask that you restrict
% the number of 'columns' to three.
%
% Because of the available 'opening page real-estate'
% we ask you to refrain from putting more than six authors
% (two rows with three columns) beneath the article title.
% More than six makes the first-page appear very cluttered indeed.
%
% Use the \alignauthor commands to handle the names
% and affiliations for an 'aesthetic maximum' of six authors.
% Add names, affiliations, addresses for
% the seventh etc. author(s) as the argument for the
% \additionalauthors command.
% These 'additional authors' will be output/set for you
% without further effort on your part as the last section in
% the body of your article BEFORE References or any Appendices.

\numberofauthors{3} %  in this sample file, there are a *total*
% of EIGHT authors. SIX appear on the 'first-page' (for formatting
% reasons) and the remaining two appear in the \additionalauthors section.
%
\author{
% You can go ahead and credit any number of authors here,
% e.g. one 'row of three' or two rows (consisting of one row of three
% and a second row of one, two or three).
%
% The command \alignauthor (no curly braces needed) should
% precede each author name, affiliation/snail-mail address and
% e-mail address. Additionally, tag each line of
% affiliation/address with \affaddr, and tag the
% e-mail address with \email.
%
% 1st. author
\alignauthor
Andre R. Mitsuoka\\%\titlenote{Dr.~Trovato insisted his name be first.}\\
       \affaddr{Universidade Federal de São Carlos - Campus Sorocaba}\\
       \affaddr{Rodovia João Leme dos Santos km 110}\\
       \affaddr{Sorocaba - SP. Brasil}\\
       \email{andrermitsuoka@gmail.com}
\and
\and  
% 2nd. author
\alignauthor
Francisco A. Guiraldelli\\%\titlenote{The secretary disavows
%any knowledge of this author's actions.}\\
       \affaddr{Universidade Federal de São Carlos - Campus Sorocaba}\\
       \affaddr{Rodovia João Leme dos Santos km 110}\\
       \affaddr{Sorocaba - SP. Brasil}\\
       \email{francisco.guiraldelli@gmail.com}
\and
\and
\and
% 3rd. author
\alignauthor
Hugo Noboru Nomura\\%\titlenote{This author is the
%one who did all the really hard work.}\\
       \affaddr{Universidade Federal de São Carlos - Campus Sorocaba}\\
       \affaddr{Rodovia João Leme dos Santos km 110}\\
       \affaddr{Sorocaba - SP. Brasil}\\
       \email{hugonomura@gmail.com}
\and  % use '\and' if you need 'another row' of author names
% 4th. author
%\alignauthor Lawrence P. Leipuner\\
       %\affaddr{Brookhaven Laboratories}\\
       %\affaddr{Brookhaven National Lab}\\
       %\affaddr{P.O. Box 5000}\\
       %\email{lleipuner@researchlabs.org}
%% 5th. author
%\alignauthor Sean Fogarty\\
       %\affaddr{NASA Ames Research Center}\\
       %\affaddr{Moffett Field}\\
       %\affaddr{California 94035}\\
       %\email{fogartys@amesres.org}
%% 6th. author
%\alignauthor Charles Palmer\\
       %\affaddr{Palmer Research Laboratories}\\
       %\affaddr{8600 Datapoint Drive}\\
       %\affaddr{San Antonio, Texas 78229}\\
       %\email{cpalmer@prl.com}
}
% There's nothing stopping you putting the seventh, eighth, etc.
% author on the opening page (as the 'third row') but we ask,
% for aesthetic reasons that you place these 'additional authors'
% in the \additional authors block, viz.
%\additionalauthors{Additional authors: John Smith (The Th{\o}rv{\"a}ld Group,
%email: {\texttt{jsmith@affiliation.org}}) and Julius P.~Kumquat
%(The Kumquat Consortium, email: {\texttt{jpkumquat@consortium.net}}).}
%\date{30 July 1999}
% Just remember to make sure that the TOTAL number of authors
% is the number that will appear on the first page PLUS the
% number that will appear in the \additionalauthors section.

\maketitle
\begin{abstract}
	Este artigo têm como finalidade descrever um sítio na web com a finalidade de ajudar as 
	pessoas a conseguirem caronas na Universidade Federal de São Carlos, campus Sorocaba. A 
	intenção é descrever suas funcionalidades, design e usabilidade, além dos príncipios e 
	necessidades de sua criação, conceitos de IHC aplicados, sistemas semelhantes e proposta 
	desse novo sistema.
\end{abstract}
% A category with the (minimum) three required fields
\category{H.6}{Human Computer Interface}{Miscellaneous}
\category{H.4}{Information Systems Applications}{Miscellaneous}
%A category including the fourth, optional field follows...
\category{D.2.8}{Software Engineering}{Metrics}[complexity measures, 
performance measures]

\terms{Theory}

\keywords{ACM proceedings, HCI}

\section{Introdução}
Ao ingressar na Universidade, o único sistema de transporte que a maioria 
dos alunos podem contar é com o transporte público. Em Sorocaba, onde 
situa-se um dos campi da Universidade Federal de São Carlos, o único 
meio de transporte público sao os ônibus que normalmente são demorados, 
devido a grande distância do centro da cidade à Universidade e em horários 
de pico estão superlotados, o quê causa desconforto e estresse aos 
usuários desse sistema.\\
Com o intuito de melhorar esta situação e percebendo que muitas pessoas
 que vão de carro à Universidade normalmente estão sozinhos ou no máximo 
 em duas pessoas e que a política do campus Sorocaba é voltada a 
 sustentabilidade, tivemos a idéia de nos aproveitar disso criando um 
sítio na internet ao qual podermos oferecer caronas ou pedir caronas 
para pessoas que se deslocam da cidade ao campus e vice-versa, o quê 
futuramente, podemos ainda extendê-lo para outros destinos diferentes 
ao da Universidade. A vantagem do sistema é que é feito tanto 
para ambientes {\it desktop} como para ambientes mobile, com o qual, 
com este último há a vantagem da computação ubíqua, ao qual podemos nos 
conectar de qualquer lugar e a qualquer momento, utilizando {\it tablets} 
e {\it smartphones} permitindo agilidade na visualização e inserção de 
caronas, além de corresponderem pela maior parte de computadores vendidos 
e ser um mercado em grande expansão. Por fim, a interface será projetada 
com a finalidade de ser a mais limpa possível e de grande usabilidade,
onde citaremos mais adiante os fundamentos de aplicação em IHC, 
dos trabalhos e sistemas semelhantes e também das vantagens desse 
sistema em relação aos outros já existentes.

\section{Fundamentos}
Relativo aos fundamentos de interface humano computador o sítio da web 
será visto como uma mídia através da qual as pessoas se comunicam umas 
com as outras e interagem entre si. Com essa finalidade, o sítio deverá 
ter cor predominante verde, passando assim segurança ao usuários e 
dando um tom agradável ao olhos.\\
A meta do sítio é prover ao usuário um ambiente simples e fácil de usar, 
dando grande importância a usabilidade, acessibilidade e comunicabilidade 
do sistema, garantindo assim, o maior número possível de usuários em 
seus diferentes níveis de conhecimento e destreza com os aparelhos 
eletrônicos. Para atingir tais critérios de usabilidade é necessário 
seguir com alguns fatores importantes ditados por {\it Nielsen (1993)}:
\begin{itemize}
	\item facilidade de aprendizado;
	\item facilidade de recordação;
	\item eficiência;
	\item segurança no uso e;
	\item satisfação do usuário.
\end{itemize}
\pagebreak


\section{Trabalhos Relacionados}
Existem vários trabalhos relacionados com caronas oferecidas não só a 
universitários como também para redes corporativas, abaixo estão 
relacionados alguns deles e o comentário relevante a cada um:

\subsection{Unicaronas}
Localizado no sítio {\tt  http://unicaronas.com.br/} não apresenta 
interface específica para dispositivos mobile. Para se cadastrar é exigido 
comprovante de matrícula de uma das Universidades vínculadas com o 
sistema além de ser obrigatório dar o número CPF. A interface de busca é 
simples e intuitiva retornando os resultados relacionados com o valor em destaque.
A parte inferior do site apresenta o logo de universidades e mídias que 
já divulgaram o site causando uma mistura de cor e fontes desagradável ao usuário. 
Com a cor principal do site sendo verde, a segurança é uma das maiores preocupações do site. 
\subsection{Carona Brasil}
Localizado no sítio {\tt  http://www.caronabrasil.com.br/} não apresenta 
interface responsiva, porém faz uma enquete para saber com quais dispositivos 
os usuários mais acessam a página na web e apesar de sua interface na cor verde 
e azul claro passarem calma e segurança aos usuários apresenta excesso de texto 
e tópicos.

\subsection{Caroneiros.com}
Localizado no sítio {\tt  http://www.caroneiros.com/web/} apresenta 
uma interface responsiva, simplificada e rápida, levando direto nos principais 
pontos de navegação, porém apresenta uma mapa de rotas que não parece ser 
funcional e sim uma simples ilustração já que o mesmo não pergunta o percurso a 
ser seguido.

\subsection{Caronetas}
Localizado no sítio {\tt http://www.caronetas.com.br/} não possui aplicação mobile, 
apesar de sua interface apresentar cores suaves e agradáveis e o cadastro ser 
simples, não ficou claro como inserir ou buscar coronas nem tampouco visualizá-las, 
dando a impressão que as coronas só são visualizadas por pessoas de uma mesma 
empresa.

\subsection{Zona Universitária}
Localizado no sítio {\tt http://www.zonauniversitaria.com.br/} não possui 
aplicação mobile, na sua página inicial apesar de agradável, já se 
percebe que é um sítio para diversos assuntos incluindo caronas, o quê 
ao entrar no site e se cadastrar é percebido um excesso de informações 
ao usuário além de não haver qualquer separação de quem procura e quem 
oferece carona.

\subsection{Tabela de Comparitivos}
Podemos verificar na tabela 1 um comparativo que mostra algumas informações 
sobre sites de caronas relativo a utilização com dispositivos mobiles:

\begin{table*}[b]
\centering
\caption{Tabela Comparativa}
\begin{tabular}{|c|c|c|c|} \hline
{\bf Sítios}
&{\bf Mobile}
&{\bf Vantagens}
&{\bf Desvantagens}\\ \hline %fim 1ª linha
%Inicio 2ª Linha
Unicaronas	
&Não			
&\parbox{5cm}{-- Interface simples e intuitiva;\\
-- Resultados com valores em destaque;\\
-- Cor agradável.}
&\parbox{5cm}{-- Difícil cadastramento;\\
-- Rodapé poluído visualmente com fontes e cores desagradáveis;}
\\ \hline
%Fim 2ª linha
%Inicio 3ª Linha
Carona Brasil
&Não 		
& -- Cores agradáveis;
& -- Apresenta excesso de textos e tópicos
\\ \hline
%Fim 3ª linha
%Inicio 4ª Linha
Caroneiros.com
&Responsivo
&Interface de uso fácil, simplicada e rápida.
&Mapa e rotas não funciona.
\\ \hline
%Fim 4ª linha
%Inicio 5ª Linha
Caronetas
&Não			
&\parbox{5cm}{\vspace{1pt}-- Interface com cores agradáveis;\\
-- Simplicidade e facilidade no cadastro.}
&\parbox{5cm}{\vspace{3pt}-- Difícil visualização e procura de caronas;\\
-- Impressão que só pessoas da mesma empresa podem visualizar caronas.\\}
\\ \hline
%Fim 5ª linha
%Inicio 6ª Linha
Zona Universitária
&Não			
&\parbox{5cm}{\vspace{1pt}-- Página inicial agradável e intuitiva.\\
-- Fácil cadastramento pelo Facebook.}
&\parbox{5cm}{\vspace{3pt}-- Não é especializado em caronas;\\
-- Excesso de informação ao usuários;\\
-- Não há separação entre quem oferece e quem precisa da carona.\\}
\\ \hline
%Fim 6ª linha
\end{tabular}
\end{table*}

\section{Proposta}
O sítio ao qual temos o intuito de implementar tem por finalidade inicialmente 
ajudar os universitários que frequentam a Ufscar Campus Sorocaba, oferecendo 
um aplicativo responsivo, ou seja, tanto para ambientes desktop, como para 
ambientes mobile como {\it tablets} e {\it smartphones} que atualmente 
correspondem pela maioria das máquinas computáveis de proprósito geral, 
num único sistema, dando o suporte para o deslocamento desses universitários.
O método de implementação tem como finalidade utilizar ferramentas ágeis e 
filosofias como {\it DRY (Don't Repeat Yourself)} e {\it KISS (Keep It Simple, Stupid)} 
oferecendo agilidade, simplicidade, usabilidade, acessibilidade e comunicabilidade.\\
Além de uma interface simples e agradável, a idéia é que o sistema seja 
compatível com browsers Google Chrome e Firefox, tanto para a versão desktop, 
quanto para a versão mobile e também seja possível utilizar o login do 
usuário utilizando o mesmo login do {\it Facebook}. Finalmente o intuito 
desse sistema que além de colaborar com os universitários e ajudá-los em sua 
locomoção diária, também é preocupar-se com o meio ambiente aplicando assim 
o conceito de sustentabilidade, criando um tripé com os conceitos ambientais, 
econômicos e sociais.
%
% The following two commands are all you need in the
% initial runs of your .tex file to
% produce the bibliography for the citations in your paper.
\bibliographystyle{abbrv}
\bibliography{sigproc}  % sigproc.bib is the name of the Bibliography in this case
% You must have a proper ".bib" file
%  and remember to run:
% latex bibtex latex latex
% to resolve all references
%
% ACM needs 'a single self-contained file'!
%
%APPENDICES are optional
%\balancecolumns
%\appendix
%%Appendix A
%\section{Headings in Appendices}
%The rules about hierarchical headings discussed above for
%the body of the article are different in the appendices.
%In the \textbf{appendix} environment, the command
%\textbf{section} is used to
%indicate the start of each Appendix, with alphabetic order
%designation (i.e. the first is A, the second B, etc.) and
%a title (if you include one).  So, if you need
%hierarchical structure
%\textit{within} an Appendix, start with \textbf{subsection} as the
%highest level. Here is an outline of the body of this
%document in Appendix-appropriate form:
%\subsection{Introduction}
%\subsection{The Body of the Paper}
%\subsubsection{Type Changes and  Special Characters}
%\subsubsection{Math Equations}
%\paragraph{Inline (In-text) Equations}
%\paragraph{Display Equations}
%\subsubsection{Citations}
%\subsubsection{Tables}
%\subsubsection{Figures}
%\subsubsection{Theorem-like Constructs}
%\subsubsection*{A Caveat for the \TeX\ Expert}
%\subsection{Conclusions}
%\subsection{Acknowledgments}
%\subsection{Additional Authors}
%This section is inserted by \LaTeX; you do not insert it.
%You just add the names and information in the
%\texttt{{\char'134}additionalauthors} command at the start
%of the document.
%\subsection{References}
%Generated by bibtex from your ~.bib file.  Run latex,
%then bibtex, then latex twice (to resolve references)
%to create the ~.bbl file.  Insert that ~.bbl file into
%the .tex source file and comment out
%the command \texttt{{\char'134}thebibliography}.
%% This next section command marks the start of
%% Appendix B, and does not continue the present hierarchy
%\section{More Help for the Hardy}
%The sig-alternate.cls file itself is chock-full of succinct
%and helpful comments.  If you consider yourself a moderately
%experienced to expert user of \LaTeX, you may find reading
%it useful but please remember not to change it.
%%\balancecolumns % GM June 2007
%% That's all folks!
\end{document}
